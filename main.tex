\documentclass{article}

\usepackage{listings}

\usepackage{parskip}
\setlength{\parskip}{0pt}

\title{Explicando Pd-Ofelia}
\author{Haruo}
\date{}

\begin{document}

\maketitle

\section{Scripts \texttt{.lua}}

A biblioteca Ofelia registra automaticamente funções nomeadas como \texttt{ofelia.*} para serem usadas como \textbf{manipuladores de mensagens} no Pure Data.
Por exemplo:

Defina uma função \texttt{ofelia.minhaMensagem()} em seu script Ofelia e associe a um objeto Pure Data. 

\subsection{\texttt{ofelia.perform()}}

Defina uma função \texttt{ofelia.perform()} para ser executada a cada ciclo DSP.

\subsection{A variável global \texttt{M}}

Imagine que você esteja escrevendo um script em Lua que será associado a um objeto do Pure Data.
Seria interessante guardar variáveis globais dentro desse script.

A variável global \texttt{M} é fornecida pelo Ofelia para o usuário armazenar e organizar configurações, variáveis e estados que necessitem ser persistentes dentro de um script Lua associado a um objeto do Pure Data.

\end{document}